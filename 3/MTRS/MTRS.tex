\documentclass{article}
\usepackage[utf8]{inputenc}

\title{Messtechnik und Regelungssystem}
\author{Felix Hofinger}
\date{April 2022}

\begin{document}

\maketitle
\newpage
\tableofcontents
\newpage

\section{Analog-Digital-Umsetzung}
Eine der Standartaufgaben der elektrischen Messtechnik besteht darin, Analoge Messsignale in entsprechende Digitalsignale (Binärzahlen) umzuwandeln.



\underline{Vorteile der Digitalen Messtechnik:}

\begin{itemize}
  \item Möglichkeit der direkten, computergestützten Weiterverarbeitung der Messdaten
  \item direkte Übernahme der Messwerte in digitale Signalverarbeitungssysteme
  \item Einfache und Langzeitsichere Speicherung
  \item unempfindlich gegen äußere Störeinflüsse
  \item keine Ablesefehler
\end{itemize}

\subsection{Abtastung (Sampling)}

Der erste Schritt bei einer Analog-Digital-Umsetzung, besteht aus der zeitlichen Abtastung des ursprünglichen Zeit und Wert kontinuierlichen Eingangssignal. Diese Abtastung wird mittels einer sogenannten Abtasthalteschaltung (Sample and Hold) vorgenommen. Durch diesen Abtastvorgang entsteht ein Zeit diskontinuierliches, aber noch Amplitudenkontinuierliches Signal. In einem weiteren Schritt wandelt der eigentliche Digitalumsetzer die zeitdiskreten Wertkontinuierlichen Abtastwerte in Form von Binärzahlen um.

\end{document}
